\chapter{Fazit}  
\label{ch:Fazit}
Zusammenfassend kann gesagt werden, dass das Ziel eines autonom spielenden Airhockeytisches nicht erreicht wurde. Zwar wurden einige gute Zwischenergebnisse hervorgebracht, aber der reale Demonstrator erfüllt seinen Zweck noch nicht.\\
Im Rahmen der Arbeit wurde die Simulation weiter entwickelt und mit einigem Trainingsaufwand wurden Agenten erzielt, die zeigen, dass die gewählten Netzwerkarchitekturen für die Aufgabe geeignet sind. In der Softwareumgebung wurde das Projekt zufriedenstellen bearbeitet, es gibt aber auch hier noch Steigerungspotenzial.\\
Die meisten Probleme gibt es noch beim Übergang auf den realen Tisch. Neben den Hardwaremängeln sind auch die Kommunikation und die Latenz durch die Anwendung aus Kapitel \ref{ch:realer demonstrator} nicht optimal. Auch hier könnte noch Aufwand betrieben werden.\\
Neben den genannten Verbesserungsmöglichkeiten bietet der Airhockey Tisch aber noch andere Ideen an, die in Folgeprojekten interessant werden könnten. Eine Möglichkeit wäre es, einen kontinuierlichen Actionspace für einen weiteren Agenten zu wähle. Das könnte dazu beitragen, dass der Roboter sich am Ende flüssiger bewegt. Auch ein Wechsel auf eine andere Roboterkinematik könnte Vorteile mit sich bringen. Ein Roboter mit Unterarm und Oberarm könnte den Pusher auf der Platte aufliegen lassen und wäre so möglicherweise weniger anfällig gegen Unebenheiten.


\newpage
\underline{Eidesstattliche Erklärung}\\\\
Hiermit versichern wir, die vorliegende Arbeit ohne fremde Hilfe und nur unter Verwendung
der angegebenen Hilfsmittel selbstständig verfasst zu haben. Alle Stellen, die wörtlich oder
sinngemäß aus veröffentlichten oder nicht veröffentlichten Arbeiten anderer Autoren entnommen sind, haben wir kenntlich gemacht.

\begin{figure} [h]
\begin{minipage}[t]{0.4\textwidth}
\vspace{0pt}
\includegraphics[scale =0.3]{images/sign_haag}

\end{minipage}
\hspace{0.1\textwidth}
\begin{minipage}[t]{0.4\textwidth}
\vspace{0pt}
\includegraphics[scale =1]{images/DB}

\end{minipage}

\vspace{5pt}
Martin Haag	\hspace{12em} Denise Baumann
\end{figure}