\chapter{Einleitung}
\label{ch:Einleitung}

In Einleitungskapitel werden die Rahmenbedingungen des Projekts erläutert.
In diesem Zuge sollen die Ausgangssituation und das Ziel genauer behandelt werden.
\section{Ausgangssituation}
\label{sect:Ausgangssituation}

Am Zentrum für maschinelles Lernen (ZML) wurde bereits in einer vorangegangenen Arbeit an einem selbst spielenden Airhockeytisch gearbeitet. Dazu wurde bereits ein Airhockeytisch mit der benötigten Hardware ausgestattet. Die Kinematik basiert dabei auf der des Roboters von jjRobots \cite{jjrob}

Das Zentrum für maschinelles Lernen (ZML) mchte einen selbstlernenden und spielenden AirHockeyRoboter entwickeln. Dafr statteten Maschinenbaustudenten einen AirHockeyTisch mit einem Kamerahalter und einem Roboter aus, welcher auf dem Konzept
der Firma jjRobots [1] basiert.
Air-Hockey ist ein hochdynamisches Geschicklichkeitsspiel, bei dem zwei Personen gegeneinander spielen. Hierfür sind die Spieler mit einem Pusher ausgestattet. Ziel des Spiels
ist es, den auf einem Luftfilm gleitenden Puck im gegnerischen Tor zu versenken. Durch
die hohen Geschwindigkeiten, welche sowohl den Puck, als auch die Pusher erreichen, besteht ein immenser Anspruch an die Mechanik, sowie an die informationsverarbeitenden
Systeme des gesamten Roboters.Das Zentrum für maschinelles Lernen (ZML) möchte einen selbstlernenden und spielenden Air-Hockey-Roboter entwickeln. Dafr statteten Maschinenbaustudenten einen AirHockey-Tisch mit einem Kamerahalter und einem Roboter aus, welcher auf dem Konzept
der Firma jjRobots [1] basiert.
Air-Hockey ist ein hochdynamisches Geschicklichkeitsspiel, bei dem zwei Personen gegeneinander spielen. Hierfür sind die Spieler mit einem Pusher ausgestattet. Ziel des Spiels
ist es, den auf einem Luftfilm gleitenden Puck im gegnerischen Tor zu versenken. Durch
die hohen Geschwindigkeiten, welche sowohl den Puck, als auch die Pusher erreichen, besteht ein immenser Anspruch an die Mechanik, sowie an die informationsverarbeitenden
Systeme des gesamten Roboters.





%\cite{sutton2018reinforcement}
\newpage